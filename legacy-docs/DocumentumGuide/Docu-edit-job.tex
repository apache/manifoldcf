% Licensed to the Apache Software Foundation (ASF) under one or more
% contributor license agreements. See the NOTICE file distributed with
% this work for additional information regarding copyright ownership.
% The ASF licenses this file to You under the Apache License, Version 2.0
% (the ``License''); you may not use this file except in compliance with
% the License. You may obtain a copy of the License at
%
% http://www.apache.org/licenses/LICENSE-2.0
%
% Unless required by applicable law or agreed to in writing, software
% distributed under the License is distributed on an ``AS IS'' BASIS,
% WITHOUT WARRANTIES OR CONDITIONS OF ANY KIND, either express or implied.
% See the License for the specific language governing permissions and
% limitations under the License.

\subsubsection{Documentum Job Options}

You must fill in the following extra fields if you choose to
configure a Documentum repository connection: 

\bigimage{Docu-edit-job-tab3}

% Licensed to the Apache Software Foundation (ASF) under one or more
% contributor license agreements. See the NOTICE file distributed with
% this work for additional information regarding copyright ownership.
% The ASF licenses this file to You under the Apache License, Version 2.0
% (the ``License''); you may not use this file except in compliance with
% the License. You may obtain a copy of the License at
%
% http://www.apache.org/licenses/LICENSE-2.0
%
% Unless required by applicable law or agreed to in writing, software
% distributed under the License is distributed on an ``AS IS'' BASIS,
% WITHOUT WARRANTIES OR CONDITIONS OF ANY KIND, either express or implied.
% See the License for the specific language governing permissions and
% limitations under the License.

\begin{itemize}
\label{scheduling}

\item \textbf{Schedule type:} Whether you want to scan every document
once or dynamically recrawl content in your repository. 

When scanning every document once, the crawler marks all documents that
have been previously crawled in this job as potentially to be deleted,
adds all seed documents to its queue and marks them as pending, processes
pending documents, marking them completed as they are ingested, and then
deleted all of the documents that were not recrawled. A document might
not be recrawled because it no longer exists, or the job specification
might have been changed to no longer include the document.

When dynamically recrawling documents, the crawler does not start by
marking all documents as potentially deletable; instead, it begins with
all of the seed documents, and continues adding to its list, periodically
re-adding the initial seed documents. If a document is removed from the
source, it will expire in the expiration interval (see below).

\item \textbf{Expiration Interval (if continuous):} The length of the
interval (in minutes) that the appliance will retain a document
crawled by this job after the document no longer appears in the
repository. After this interval, the missing document will be removed
from the appliance's index and archive. Leave the expiration interval
blank to keep missing documents indexed in GTS.

\item \textbf{Recrawl interval:} If you are dynamically recrawling
documents, how long, in minutes, the crawler should wait before
crawling documents a second time.

\item \textbf{Reseed interval:} If you are dynamically recrawling
documents, how long, in minutes, the crawler should wait before
looking for new documents to crawl. \ifMeridioGuide This connector
identifies all documents for ingestion through seeding; if the reseed
interval is infinite, the job will not ingest documents placed in the
repository during run time. (The job automatically reseeds whenever it
is started.) The default interval of 60 minutes is an appropriate
reseed rate. \fi \ifFilenetGuide This connector identifies documents
for ingestion during seeding. If you change the document inclusion
criteria, reseeding is required to identify new documents. Similarly,
documents placed in the repository while the job is running will not
be identified until the crawl is reseeded.  (The job automatically
reseeds whenever it is started.) The default interval of 60 minutes is
an appropriate reseed rate. \fi

\item \textbf{Scheduled time:} Allows you to define a time you wish
the job to run using a series of selection boxes. The first box refers
to the day of the week you wish the job to run, with an option to have
the job run any day of the week. The second box allows you to select
the start hour, with an option to start the job at any hour. The third
box allows you to specify which minute after the hour that you wish
the job to start. The fourth box allows you to specify what months of
the year you wish the job to run, with an option for the job to run
any month. The last box allows you to specify the day of the month you
wish the job to start, including any day of month.


You can scroll through each of the five boxes in this setting using
the arrow keys on your keyboard or by using the scroll bar on the
right side of the box.  If you want to select more than one value,
hold down control as you scroll and click the values that you want to
select. This allows you to define multiple windows with the same
length, for example by selecting Monday, Wednesday, and Friday at the
same time.

\item \textbf{Maximum run time:} The longest you will allow the job to
run, in minutes. For example, if you want to start a job at 2 AM but
force it to stop at 8 AM so that users have access to the repository,
you should set this value to 360 minutes. If the job is not complete by the
end time, documents that have already been found will be indexed, and
the rest of the crawl will continue at the beginning of the next
schedule interval. 

When you have defined the scheduled time and assigned a maximum run
time, click on the ``Add Scheduled Time'' button. A new schedule box
will appear below the scheduled time, allowing you to create
additional scheduled run times.

Here is a sample schedule for a job that will run every
Monday from 2 am to 6 am:

\begin{changemargin}{-.3in}{0in} 
\includegraphics[width=300pt]{sample-schedule}
\end{changemargin}

If you do not have at least one scheduled time, the job will
only run when run manually (see page \pageref{ManageJobs}), and will
not automatically update the index on the appliance based on changes
to the repository.

You can remove a scheduled time by clicking the ``Remove Schedule''
button.

\end{itemize}


\includegraphics[width=300pt]{Docu-edit-job-tab4}

\begin{itemize}

\item \textbf{Paths:} The directory paths in your Documentum
repository from which you want your crawl to start. You can specify
one or more directory paths. If you do not specify directory paths,
the job will crawl all directories on your Documentum repository. You
can build directory paths by selecting individual directories. To
start, select the base directory of the path you wish to create, then
click the ``+'' button. A new selection box will appear with the
directories contained by that parent directory. Select a directory at
that level and click the ``+'' button. Continue building your
directory path in this fashion. When it's complete, click ``Add''. A
new selection box will appear beneath the added directory path. You
can continue to add more directory paths to your list. To remove a
directory path from the list, click the ``Delete'' button next to it.


\end{itemize}

\includegraphics[width=300pt]{Docu-edit-job-tab5}

\begin{itemize}

\item \textbf{Document Types:} Select here the Documentum document
subclasses you wish to include in your job. Simply check the
subclasses you wish to include. The list of document subclasses is
generated by your Documentum repository based on existing document
subclasses.

\item \textbf{Metadata:} The Documentum repository stores various
metadata information about documents in its index.  You can select
those metadata fields here and have them sent along with the files you
index as metadata.  Each document subclass has its own selection box,
listing the metadata fields available for that particular
subclass. You can select ranges using the Shift key and make multiple
selections using the Ctrl key.  Metadata will not be geographically
parsed or used to create the index on the MetaCarta appliance;
however, with the standard MetaCarta Search APIs, you can construct
searches based specifically on this metadata. For more information on
the SOAP Search API, please see the \documentref{MetaCarta SOAP Search
API Guide}. For more information on the JSON Search API and KML Search
API, please see the \documentref{MetaCarta Guide to Web Services
Search APIs}.

\end{itemize}

\includegraphics[width=300pt]{Docu-edit-job-tab6}

\begin{itemize}

\item \textbf{Content Types:} Here you can select the Documentum
content types you wish this job include. This list of Documentum
content types is generated by your Documentum repository based on the
content types present in the system. These content types will not
correspond directly with the filetypes supported by MetaCarta;
however, the supported filetypes provide a rough guideline for the
types of content that can be ingested.

% Licensed to the Apache Software Foundation (ASF) under one or more
% contributor license agreements. See the NOTICE file distributed with
% this work for additional information regarding copyright ownership.
% The ASF licenses this file to You under the Apache License, Version 2.0
% (the ``License''); you may not use this file except in compliance with
% the License. You may obtain a copy of the License at
%
% http://www.apache.org/licenses/LICENSE-2.0
%
% Unless required by applicable law or agreed to in writing, software
% distributed under the License is distributed on an ``AS IS'' BASIS,
% WITHOUT WARRANTIES OR CONDITIONS OF ANY KIND, either express or implied.
% See the License for the specific language governing permissions and
% limitations under the License.

MetaCarta GTS currently supports the following filetypes:

\begin{itemize}
\item ASCII Text Files with or without extensions (.txt, etc...)
\item HTML Documents (.htm, .html)
\item Adobe\circler\ Acrobat\circler\ files (.pdf)
\item Adobe PostScript\circler\ files (.ps)
\item Microsoft\circler\ Word\circler\ documents (.doc)
\item Microsoft Excel\circler\ spreadsheets (.xls)
\item Microsoft PowerPoint\circler\ presentations (.ppt)
\item Rich Text Format documents (.rtf)
\end{itemize}

\note{Documents larger than 50MB are converted to plaintext and then truncated to 50MB.} 



Some content types are easily matched to filetypes; for example the
content type \command{html} corresponds to HTML documents, and the
content type \command{pdf} corresponds to Adobe Acrobat
files. Other relationships may not be as clear. The content types
\command{msw}, \command{msw3}, \command{msw6}, \command{msw8},
\command{mwsm}, \command{mswm1}, and \command{msww} all correspond to
Microsoft Word documents, while the content type
\command{doc} corresponds to an Interleaf\circler\ 3.x or 4.x
file. Your Documentum installation includes a full list of the default
content types and their formats at
\dirpath{%DM_HOME%/install/tools/formats.csv}. See your Documentum
documentation for more details.

If documents of a given content type cannot be ingested they will not
be indexed by the appliance.

\end{itemize}

\includegraphics[width=300pt]{Docu-edit-job-tab7}

\begin{itemize}

\item \textbf{Content Length:} The maximum file size, in bytes, that
you wish this job to crawl. Files larger than this size will be
skipped. The default maximum content length is unlimited. 

\end{itemize}


\includegraphics[width=300pt]{Docu-edit-job-tab8}

\begin{itemize}

\item \textbf{Security:} Use this option to enable or disable document
security. If you choose to enable security, user permissions will be
ingested with documents, while if you choose to disable security,
documents will be ingested without permissions.

\item \textbf{Access Tokens:} \label{ForceACL} If you wish to specify
your own ACLs for files ingested through this job, you can specify
them here. You should use this option if you selected ``Standard
(Kerberos)'' as the authority connection for your repository
connection and you are choosing to enable security. Simply enter one
or more ACL identifiers into the field and click the ``Add''
button. The ACL identifiers will appear in a list. You can continue to
add more ACL identifiers using the ``Add'' button, or remove them
using the ``Delete'' button that appears next to each ACL identifier.

\end{itemize}

\includegraphics[width=300pt]{Docu-edit-job-tab9}

\begin{itemize}

\item \textbf{Path Attribute name:} The name for the metadata field
representing path attributes. This should be a recognizable name
distinct from any of the metadata fields already associated with any
document type being ingested. It may be useful to have the metadata
field containing path attributes have the same name across jobs.

\item \textbf{Path-value mapping:}The regular expressions and
substitutions that you want to use to collect information from the
Documentum file path. You can construct one or more regular
expressions. In the example shown, there are two expressions. The
first, \verb+.*/(.*)/(.*)/.*+ to \verb+$(1) $(2)+, would change the
directory path ``Project/Folder\_1/Folder\_2/ Filename'' into
``Folder\_1 Folder\_2.'' The second, \_ to a space, would then be
applied to turn the metadata into ``Folder 1 Folder 2.'' It is
important to allow more than one transform so that you can, if
necessary, extract text data and then parse the extracted data. The
end result of the last transform will be ingested as the value of the
metadata attribute defined previously. For more information on regular
expressions, see the note on page \pageref{regex}.

\end{itemize}


After entering this information, you will be taken to the status page
for this job:

\includegraphics[width=300pt]{Docu-view-job-status}

