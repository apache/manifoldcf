% Licensed to the Apache Software Foundation (ASF) under one or more
% contributor license agreements. See the NOTICE file distributed with
% this work for additional information regarding copyright ownership.
% The ASF licenses this file to You under the Apache License, Version 2.0
% (the ``License''); you may not use this file except in compliance with
% the License. You may obtain a copy of the License at
%
% http://www.apache.org/licenses/LICENSE-2.0
%
% Unless required by applicable law or agreed to in writing, software
% distributed under the License is distributed on an ``AS IS'' BASIS,
% WITHOUT WARRANTIES OR CONDITIONS OF ANY KIND, either express or implied.
% See the License for the specific language governing permissions and
% limitations under the License.

\begin{changemargin}{1.5in}{0in}

\section{Overview}

The MetaCarta GTS appliance indexes documents and allows users to
search these documents based on both keywords and geographic
references. The MetaCarta Sharepoint Connector allows system
administrators to configure connections and define jobs to ingest
documents from Microsoft\circler Sharepoint\circler repositories. This
connector supports Microsoft Windows Sharepoint Services\circler 2.0
(2003) and 3.0 (2007).

This document specifies the means for connecting to Sharepoint
repositories, indexing files from these repositories, and maintaining
connections to these repositories.

\subsection{Assumptions}

This document assumes you have a basic level of familiarity with GTS
appliance administration. This document also assumes that you have a
basic understanding of the repositories to which you are trying to
connect. If you need more information about the MetaCarta GTS
appliance, please read the \documentref{MetaCarta GTS Administrator's
Guide} stored on the appliance at
\dirpath{/usr/share/doc/metacarta/AdminGuide.pdf}.

For more information about Sharepoint, contact your Sharepoint
administrator or view Microsoft's documentation at
\url{office.microsoft.com/en-us/}\linebreak\url{sharepointportaladmin/default.aspx}.

Throughout this document, we assume that your appliance is named \\
\url{metacarta.example.com}.

\section{Installation}

The Sharepoint Connector must be used with the MetaCarta Connector
Framework, described in the \documentref{Metacarta Connector
Guide}. If you already have the Connector Framework installed, you can
install the Sharepoint Connector.  The Sharepoint Connector is
available as a field-test addon ISO file from from Metacarta. Use the
following steps to install the Sharepoint Connector to your appliance:

\begin{enumerate}

\item Confirm that the system is running correctly using
\command{check\_system\_health}.

\item Copy the disc image and md5sum supplied by MetaCarta to the
\dirpath{/isos} directory on your appliance. Rename the ISO file to
\dirpath{addon-shp-conn.iso} when you copy it to your appliance.

\item Check the md5sum of the ISO file using the \command{md5sum} command.

\item Install the iso using the following command:

\command{upgrade\_control install /isos/addon-shp-conn.iso}

The appliance will reboot once the installation is complete.

\item Upgrade your license file, if necessary, as described in
\documentref{MetaCarta Appliance Administrator's Guide}. You may need
to contact MetaCarta Customer Service to obtain the appropriate
license file. (Please see page \pageref{SupportContact} for contact
information.)
 
\end{enumerate}

\subsection{MetaCarta Sharepoint Web Service Installation}\label{SWService}

If you are using the Sharepoint Connector to connect
to a Sharepoint Services 3.0 server, you need to
install the Sharepoint Web Service on that server. The
Sharepoint Web Service installer is contained in the archive
\dirpath{/usr/share/metacarta/MetaCartaSharePointWebServiceInstaller.zip},
which the Sharepoint\\ Connector installation has placed on your
appliance. Copy this archive to your Sharepoint server and then unpack
it. Read the file \dirpath{Installation} \dirpath{Readme.txt} from the
unpacked archive, and follow the directions in the file. This package
also contains several executable files allowing you to install, upgrade,
or remove the Sharepoint Web Service. The detailed commands for each
are described in the \dirpath{Installation} \dirpath{Readme.txt} file.

You do not need to install the Sharepoint Web Service if you are
connecting to a Sharepoint Services 2.0 server.

\section{Configuration}

\subsection{Access to the Connector}

The administrator to the Sharepoint Connector % Licensed to the Apache Software Foundation (ASF) under one or more
% contributor license agreements. See the NOTICE file distributed with
% this work for additional information regarding copyright ownership.
% The ASF licenses this file to You under the Apache License, Version 2.0
% (the ``License''); you may not use this file except in compliance with
% the License. You may obtain a copy of the License at
%
% http://www.apache.org/licenses/LICENSE-2.0
%
% Unless required by applicable law or agreed to in writing, software
% distributed under the License is distributed on an ``AS IS'' BASIS,
% WITHOUT WARRANTIES OR CONDITIONS OF ANY KIND, either express or implied.
% See the License for the specific language governing permissions and
% limitations under the License.

must have
access to the web interface at
\url{http://metacarta.example.com/crawler/}. In the default appliance
security setup, you must have a Basic Authentication account
configured for access to the Connector web interface at
\url{http://metacarta.example.com/crawler/}.  If you are not an appliance
administrator, please ask the appliance administrator to give you such
an account.

An appliance administrator can create an account with access to the
ingestion interface (in this case, username {\tt fred} and password
{\tt ginger}) by running the following command on the appliance:

\begin{console}
metacarta:\~{}\$ basic\_auth\_control add ingest\_users fred:ginger 
\end{console}

Depending on how you have configured authentication using the
\command{auth\_control} tool, you may need to make changes other than
adding yourself to the ingest\_users group. For more information on
security configuration and \command{auth\_control}, see the Security
Administration section of the \documentref{MetaCarta Appliance
Administrator's Guide}.


\section{Collecting Documents From Sharepoint Repositories} % Retitle this, yo.

The Connector Framework manages retrieving documents from different
feeds and repositories through \emph{jobs}. Jobs can be scheduled to
run regularly; each job connects to a single repository using a
particular set of credentials. Each job is tied to a \emph{repository
connection}. Repository connections contain information allowing the
connector framework to connect to a given repository. A repository
connection may be tied to an \emph{authority connection}, which
manages document security. While using the Sharepoint Connector to
connect to a Sharepoint repository, no authority connection is needed;
you can simply create a repository connection.


\subsection{Creating Repository Connections}

In order to create jobs to ingest documents, you need to create a
repository connection. To do so, click ``List Repository
Connections'' on the sidebar menu. Then, when presented with the list
of repository connections, click ``Add a new connection.'' You will
see the following two tabs:

\includegraphics[width=300pt]{shp-edit-repository-tab1}

\includegraphics[width=300pt]{shp-edit-repository-tab2}



\input{sidebar}



Now you must provide a name, description, and connector type for your
new repository connection. The name should be unique, as you will use
it to select this connection later when defining jobs. The description
should explain the repository connection to you or another
administrator.  The connector type is the type of repository from
which you will get documents, in this case a SharepointConnector. The
authority type is the type of authority from which you will get
authorization information. The Sharepoint Connector does not use an
administrator-created authority; you should simply select ``Standard
(Kerberos)'' here.

Once you have filled in those tabs, click Continue to be taken to the
repository-specific options.

% Licensed to the Apache Software Foundation (ASF) under one or more
% contributor license agreements. See the NOTICE file distributed with
% this work for additional information regarding copyright ownership.
% The ASF licenses this file to You under the Apache License, Version 2.0
% (the ``License''); you may not use this file except in compliance with
% the License. You may obtain a copy of the License at
%
% http://www.apache.org/licenses/LICENSE-2.0
%
% Unless required by applicable law or agreed to in writing, software
% distributed under the License is distributed on an ``AS IS'' BASIS,
% WITHOUT WARRANTIES OR CONDITIONS OF ANY KIND, either express or implied.
% See the License for the specific language governing permissions and
% limitations under the License.

\subsubsection{Configuring a Sharepoint Connector}

You must fill in the following tabs if you are configuring a
Sharepoint Connector:

\includegraphics[width=300pt]{shp-edit-repository-tab3}

\begin{itemize}

\item \textbf{Max connections (per JVM):} Here you can set the maximum
number of connections to your repository.  \ifCombinedConnectorGuide
The maximum number of connections per JVM is important for three
reasons; licensing, appliance resources, and the possiblity of
overwhelming the ingestion interface. For a more complete explanation,
see the Max Connections item on page \pageref{maxrepocon}.\fi

\ifJDBCGuide
The maximum number of connections per JVM is important for several
reasons.  First, the number of connections may impact the resources
available on your Sharepoint server.

Second, the number of connections may impact the resources
available on the appliance. If the connector framework is slowing down
your appliance, lowering this number should help.

Third, only ten document streams can be processed by the appliance
at one time.  If you are also using other repository connectors or
the \command{ingest} command on the appliance, you should reduce this
number to prevent contention for the Ingestion interface. The Sharepoint
Connector will never overwhelm the interface on its own, but when other
applications are also using the ingestion interface, it may be best to
set the number of repository connections to five or even fewer.
\fi


\item \textbf{Throttling:} Here you can set a maximum document fetch
rate for the repository connection.  The maximum fetch rate allows you
to set three things: Expression, description, and fetches per minute.

In the Sharepoint Connector, the expression field can be used to
create throttle groups based on server name. Put the server's fully
qualified domain name in the expression field, and set the maximum
average number of document fetches per minute. Once you have set that,
click Add. Creating a throttling group with a blank expression will
throttle all servers together.

\end{itemize}

\includegraphics[width=300pt]{shp-edit-repository-tab4}

To fill out this tab, you will need information about the Sharepoint
server with which you wish to connect. One method of determining
protocol, name, and site path information is to browse to the
Sharepoint root or managed site that you wish to crawl, and decompose
the URL into the necessary information. For example, if your
Sharepoint site URL is
\url{http://server.example.com/Sample/Documents/default.aspx}, the
appropriate protocol is \texttt{http}, the server name is
\texttt{server.example.com}, and the site path is
\texttt{/Sample/Documents}. The site path should not include subsite information; subsite information is included during crawler job creation. You may need
to ask your Sharepoint administrator to provide other information.

\begin{itemize}

\item \textbf{Server protocol:} Select ``http'' or ``https'' using the drop down box. The protocol appears at the beginning of the Sharepoint site URL.

\item \textbf{Server name:} The name of your Sharepoint server.

\item \textbf{Server port:} The port to use to connect to the server. You should ask your Sharepoint administrator for the correct port.

\item \textbf{Site path:} The site path for the Sharepoint site on your server that you wish to crawl. This may be blank if you are crawling the root site of your server.

\item \textbf{User name:} This should be the Active Domain user name used by your MetaCarta appliance, in the format \texttt{domain$\backslash$username}.

\item \textbf{Password:} The password corresponding to the user name given.

\item \textbf{SSL certificate list:}
If you specified ``https'' as the server protocol above, you may need to
upload appropriate SSL certificates or certificate authorities here. The
repository connection will need certificates similar to those used to
connect to your Sharepoint site using an Internet browser.

If the certificate authority used to sign your server certificate is a
well-known authority, you will not need to upload a certificate
here. The appliance will automatically accept a certificate from the
server. If the server certificate is signed by an unknown authority,
you should upload the authority's certificate. In some cases, the
authority may be unavailable. In this case, you can upload the
server-side certificate itself. Server-side certificate changes may
require you to upload newer versions of this certificate if you use
this option.

\end{itemize}

\note{If the server you specify on this tab is running Sharepoint Services 3.0 you must install the MetaCarta Sharepoint Web Service on that server. See page \pageref{SWService} for instructions on installing the Sharepoint Web Service.}


\includegraphics[width=300pt]{shp-view-repo-conn-status}

In this example (which does not contain accurate information for any
Sharepoint Connector), the Connection Status is ``Connection
working.'' If connetion has failed, the status message may contain
information about the failure.  If you see an error message, you most
likely have incorrectly entered one of the fields, and should click
``Edit'' to fix the data. If you have entered everything as you
intended, please inform your database administrator; you may not have
been given the correct information.


\subsection{Creating and Running Jobs}

To run a job, click ``Status and Job Management'' on the sidebar menu.
You can run or edit existing jobs from this menu.

To create a new job, click ``List All Jobs'' on the sidebar menu. Then, when
presented with the list of current jobs, click ``Add a new job.'' You
will be presented with two tabs, in which you must fill in the following
information:

\includegraphics[width=300pt]{shp-edit-job-tab1}

\begin{itemize}

\item \textbf{Name:} The name of the job. You will use this to identify
the job later.

\item \textbf{Collection:} The collection name metadata for all
documents in this job. End users can use this name to select the set
of documents in this job. For more information on collection name
metadata, please see the \documentref{MetaCarta GTS Administrator's
Guide}.

\item \textbf{Document template:} \input{doctemplate}

\end{itemize}

\includegraphics[width=300pt]{shp-edit-job-tab2}

\begin{itemize}

\item \textbf{Connection:} The name of the repository connection you
wish to use for this job. You select this from the list of repository
connections you have already made. You may have more than one job use
the same repository connection, but if you have two jobs crawl the same
documents, the documents will have the metadata and collection name
associated with whatever job crawled the document most recently. This
will cause unpredictable results when searching those collections,
searching for those documents, or trying to delete those collections.
We recommend never crawling the same document in two different jobs.

\item \textbf{Start method:} Whether you want to start this job the next
time jobs are scheduled to run (``Start when schedule window starts''),
immediately after you finish defining it (``Start even inside a schedule
window''), or not at all (``Don't automatically start this job'').

\item \textbf{Priority:} From 1 (highest) to 10 (lowest), the priority
this crawl should have if it must compete for resources with other
crawls on the appliance. You should not need to change this unless you
are running more than one crawl at the same time; if you are, assign a
higher priority to the crawls whose documents you want to be processed
preferentially before documents from other jobs.

\end{itemize}

After filling in those options, click ``Continue'' and you will be
presented with two additional repository-specific tabs.

% Licensed to the Apache Software Foundation (ASF) under one or more
% contributor license agreements. See the NOTICE file distributed with
% this work for additional information regarding copyright ownership.
% The ASF licenses this file to You under the Apache License, Version 2.0
% (the ``License''); you may not use this file except in compliance with
% the License. You may obtain a copy of the License at
%
% http://www.apache.org/licenses/LICENSE-2.0
%
% Unless required by applicable law or agreed to in writing, software
% distributed under the License is distributed on an ``AS IS'' BASIS,
% WITHOUT WARRANTIES OR CONDITIONS OF ANY KIND, either express or implied.
% See the License for the specific language governing permissions and
% limitations under the License.

\subsubsection{Sharepoint Job Options}

You must fill in six more tabs to configure a Sharepoint job.

\bigimage{shp-edit-job-tab3}

\ifJDBCGuide
% Licensed to the Apache Software Foundation (ASF) under one or more
% contributor license agreements. See the NOTICE file distributed with
% this work for additional information regarding copyright ownership.
% The ASF licenses this file to You under the Apache License, Version 2.0
% (the ``License''); you may not use this file except in compliance with
% the License. You may obtain a copy of the License at
%
% http://www.apache.org/licenses/LICENSE-2.0
%
% Unless required by applicable law or agreed to in writing, software
% distributed under the License is distributed on an ``AS IS'' BASIS,
% WITHOUT WARRANTIES OR CONDITIONS OF ANY KIND, either express or implied.
% See the License for the specific language governing permissions and
% limitations under the License.

\begin{itemize}
\label{scheduling}

\item \textbf{Schedule type:} Whether you want to scan every document
once or dynamically recrawl content in your repository. 

When scanning every document once, the crawler marks all documents that
have been previously crawled in this job as potentially to be deleted,
adds all seed documents to its queue and marks them as pending, processes
pending documents, marking them completed as they are ingested, and then
deleted all of the documents that were not recrawled. A document might
not be recrawled because it no longer exists, or the job specification
might have been changed to no longer include the document.

When dynamically recrawling documents, the crawler does not start by
marking all documents as potentially deletable; instead, it begins with
all of the seed documents, and continues adding to its list, periodically
re-adding the initial seed documents. If a document is removed from the
source, it will expire in the expiration interval (see below).

\item \textbf{Expiration Interval (if continuous):} The length of the
interval (in minutes) that the appliance will retain a document
crawled by this job after the document no longer appears in the
repository. After this interval, the missing document will be removed
from the appliance's index and archive. Leave the expiration interval
blank to keep missing documents indexed in GTS.

\item \textbf{Recrawl interval:} If you are dynamically recrawling
documents, how long, in minutes, the crawler should wait before
crawling documents a second time.

\item \textbf{Reseed interval:} If you are dynamically recrawling
documents, how long, in minutes, the crawler should wait before
looking for new documents to crawl. \ifMeridioGuide This connector
identifies all documents for ingestion through seeding; if the reseed
interval is infinite, the job will not ingest documents placed in the
repository during run time. (The job automatically reseeds whenever it
is started.) The default interval of 60 minutes is an appropriate
reseed rate. \fi \ifFilenetGuide This connector identifies documents
for ingestion during seeding. If you change the document inclusion
criteria, reseeding is required to identify new documents. Similarly,
documents placed in the repository while the job is running will not
be identified until the crawl is reseeded.  (The job automatically
reseeds whenever it is started.) The default interval of 60 minutes is
an appropriate reseed rate. \fi

\item \textbf{Scheduled time:} Allows you to define a time you wish
the job to run using a series of selection boxes. The first box refers
to the day of the week you wish the job to run, with an option to have
the job run any day of the week. The second box allows you to select
the start hour, with an option to start the job at any hour. The third
box allows you to specify which minute after the hour that you wish
the job to start. The fourth box allows you to specify what months of
the year you wish the job to run, with an option for the job to run
any month. The last box allows you to specify the day of the month you
wish the job to start, including any day of month.


You can scroll through each of the five boxes in this setting using
the arrow keys on your keyboard or by using the scroll bar on the
right side of the box.  If you want to select more than one value,
hold down control as you scroll and click the values that you want to
select. This allows you to define multiple windows with the same
length, for example by selecting Monday, Wednesday, and Friday at the
same time.

\item \textbf{Maximum run time:} The longest you will allow the job to
run, in minutes. For example, if you want to start a job at 2 AM but
force it to stop at 8 AM so that users have access to the repository,
you should set this value to 360 minutes. If the job is not complete by the
end time, documents that have already been found will be indexed, and
the rest of the crawl will continue at the beginning of the next
schedule interval. 

When you have defined the scheduled time and assigned a maximum run
time, click on the ``Add Scheduled Time'' button. A new schedule box
will appear below the scheduled time, allowing you to create
additional scheduled run times.

Here is a sample schedule for a job that will run every
Monday from 2 am to 6 am:

\begin{changemargin}{-.3in}{0in} 
\includegraphics[width=300pt]{sample-schedule}
\end{changemargin}

If you do not have at least one scheduled time, the job will
only run when run manually (see page \pageref{ManageJobs}), and will
not automatically update the index on the appliance based on changes
to the repository.

You can remove a scheduled time by clicking the ``Remove Schedule''
button.

\end{itemize}

\fi

\ifCombinedConnectorGuide
This tab presents scheduling options. Here you can generate one or
more scheduled run times for the job. For a complete description of
the scheduling options, see the description starting on page
\pageref{scheduling}.
\fi

\bigimage{shp-edit-job-tab4}

\begin{itemize}


% I want to do more with the second and third sentences there, but
% I am not seeing a better way either at the moment. :/
\item \textbf{Path rules:} \label{pathrules}
Path rules allow you to include or exclude
subsites, libraries, and files from your crawl. From the point of view
of this tool, a Sharepoint file path has two parts. The first is the
server and site path specified during the creation of the repository
connection, and the second is the location of the file inside that
site. For example, the file path of a given file may be

\begin{changemargin}{-1.5in}{0in}\dirpath{http://server.example.com/SampleSite/Subsite/Library/Folder/file}
\end{changemargin}

You already specified the first part of the path,
\dirpath{http://server.example.com/SampleSite}, when you created the
repository connection. You must now specify subsite, library, folder,
and file information in the path tool. At minimum, you must specify
containing library and file name information in order to crawl any
documents.

% Malima, I think I am tightening this up, can you make sure I am not
% changing any facts by accident? Thanks!

The paths tool has
three sections. At the top, the paths tool displays inclusion and 
exclusion rules that you have already created and added to your
crawl. At the left of the rules are buttons that allow you to delete a
rule, insert a new rule at any place in the list, and add a new rule
to the end of the list. Inclusion/exclusion rules are applied in
order; rules at the top of the list supersede rules further down the
list.

The second section displays the path rule currently being created. The
path match is displayed, as is the rule type. The rule type indicates
if the rule is to be applied to a site, a library, or files. If the
rule includes a text match portion, you will need to select a type. At
the right of this section, you can choose whether the rule you are
creating will be an inclusion or exclusion rule. To add or insert the
displayed rule into the list of path rules, use the appropriate
``Add new rule'' or ``Insert new rule'' button to the left of the
rules list in the first section.
%What does "The appropriate button" mean?  I couldn't tell.

The third section allows you to select sites and libraries and to add
wildcard text matching that will be used in a rule. The
crawler automatically discovers subsites and libraries when the
connection used for a job is working properly. To select a subsite or library
to add to the path rule, simply select it from the appropriate
selection box and click the accompanying add button. To create a match
expression, enter a wildcard match expression in the text box at the
right, and click the ``Add Text'' button. When you add a match
expression, you must then select the match type in the second section
of the path tool. The wildcard expression you entered can be used to
match files, sites, or libraries. To change the rule under
construction, you can click the ``Reset Path'' button to clear the
path rule, or, if applicable, the minus button to remove the last step
of the path rule. The minus button only appears when the last step of
the path rule shown is a Site or Library selection.


\end{itemize}


Initially, no path rules are in place, and, thus, no documents will be
crawled by the job. To include everything in your system, you will need to create a set of rules that include all sites, libraries, and files.

At the top of the list, create a rule that includes all sites. In the
third section of the tool, enter the match expression \texttt{*} in
the text field and click the ``Add text'' button. In the second
selection, select the type ``Site'' and the ``include'' indicator,
then click the ``Add new rule'' button in the top section. This will
add the new rule including all subsites.

Next, create a new rule including all libraries by using the match
expression \texttt{*}, the ``Library'' type, and the ``include''
indicator. Use the ``Add new rule'' button to add it to the bottom of
the list.

Finally create a file rule using, \texttt{*}, ``File'', and
``include'' and add it to the bottom of the list.

This set of rules, shown below, will include all subsites, libraries, and files
accesible from the base site of the crawl.

\bigimage{shp-edit-job-tab4a}

Wildcard expression matching can be very useful for creating more
complicated rules. For example, you might wish to create a rule that
excluded all libraries including the word ``Test'' from a
given subsite, ``SubSite.'' Using the third section of the paths
tool, you would first select ``SubSite'' in the site selection box
and click ``Add Site.'' Then you would type the expression
\texttt{*Test*} in the text box and click the ``Add Text'' button. 
You would select the type ``Library'' and action ``exclude'' in the
second section, as shown below:

\includegraphics[width=300pt]{shp-edit-job-tab4b}


Click the ``Insert New Rule'' button at the top of the list of rules. This
creates a new rule that excludes
any Library from the subsite ``SubSite'' including the expression
``test'' or ``Test'' in its name. Because this rule is at the top of the list, it will superceed the inclusion rules further down in the list.

\includegraphics[width=300pt]{shp-edit-job-tab4c}



\includegraphics[width=300pt]{shp-edit-job-tab5}

\begin{itemize}

\item \textbf{Security:} Select ``Enabled'' to use Active Directory permissions with the ingested documents. You can configure custom permissions using the ``Access Tokens'' tools, or accept the ACLs as given in the Sharepoint repository. Select ``Disabled'' to allow all document search users to access to the files ingested by this job.

\item \textbf{Access Tokens:} This field allows you to create custom ACLs for the files ingested by this job. Enter the SID of a user or group that you wish to add to the custom ACLs for documents from this job, then click the ``Add'' button. You can continue to add more SIDs. These SIDs will appear in a list. Click the ``Delete'' button next to any SID to remove it from the list.

\end{itemize}


\bigimage{shp-edit-job-tab6}

\begin{itemize}

\item \textbf{Metadata rules:} Using this tool, you can create inclusion and exclusion rules to determine what metadata from the Sharepoint server is ingested with documents crawled by this job. This tool operates in a similar manner to the path rules tool, discussed on page \pageref{pathrules}. The first section displays the list of rules and allows you to delete current rules and add or insert new rules. The second section shows the rule you are currently building. The third section contains tools to insert site, library, and regular expression matching information into the current rule.

To create a metadata rule, you create an inclusion or exclusion rule
in the same fashion as a path rule. In the case of exclusion, all
subsites, libraries, or files specified by your path rule will be
ingested without metadata. For inclusion, you can select to have all
metadata included with the subsites, libraries, or files specified by
your rule, or, in the case of libraries, select individual metadata
fields to be ingested. Highlight individual fields by clicking,
multiple fields by clicking while holding the ``Ctrl'' key, or ranges
of fields by clicking while holding the ``Shift'' key.

As in the case of the path rules, the rules are applied in order from
the top of the list to the bottom. By default, no system metadata is
included. To create a rule to include all metadata, select ``include''
and check the ``Include all metadata'' box in the second section of
the tool, then click the ``Add new rule'' button. The default path is
the base path, so the new rule will instruct the crawler to collect
all metadata for all files crawled through this job.

\item \textbf{Path metadata:} This tool allows you to create a metadata attribute based on URL path data. You can use regular expressions to manipulate the URL. First, enter the name for your path metadata attribute. In the example shown, the attribute is called ``pathdata''. If you do not provide any regular expression mappings, the path metadata will just be the document's path information, not including the path information from the base site of your crawl. For example, the path metadata for a file ``sample'' in a library ``Example'' on the base site of your crawl would be ``/Example/sample''. In this case, all files will have unique values for this metadata attribute.

In the example shown above, the path metadata is manipulated by using
the match expression \texttt{(.*)/(.*)\$} and the replace expression
\texttt{\$(1)}. This mapping strips the file name from the path
metadata, leaving just the site, library, and folder information in
the attribute. For more information on regular expressions, see the
descriptions and note in the path rules section starting on page
\pageref{pathrules}.

To add a mapping, enter a match expression in ``Match regexp'' field
and the replacement expression in the ``Replace string'' field, then
click the ``Add Path Mapping'' button. The mappings are performed in
order from the top of the list down. To remove a mapping from the
list, simply click the ``Delete Path Mapping'' button to its left.


\note{If you are not familiar with regular expressions, there
are a variety of tutorials available on the web, including
\url{http://gnosis.cx/publish/}\linebreak\url{programming/regular_expressions.html}
and \url{http://perldoc.}\linebreak\url{perl.org/perlrequick.html}. If
you still have difficulty with these settings, please contact Customer
Support (see page \pageref{SupportContact}).}


\end{itemize}

After entering this information, you will be taken to the status page
for this job:

\includegraphics[width=300pt]{shp-view-job-status}


\subsection{\label{ManageJobs}Status and Job Management}

You can then look at the status of your job by clicking ``Status and 
Job Management'' on the sidebar. You will see a list of one or more jobs
much like this one:

\includegraphics[width=300pt]{shp-jobs-list}

You can start any crawl you like immediately from this interface by
clicking ``Start'' next to the name of the crawl. This interface also
allows you to see how many documents have been crawled; this information
may help you structure and plan future crawls.

\note{Refresh this page by clicking the ``Refresh'' link at the bottom
of the page, not by clicking your browser's reload button.}

\section{Reports}

The Connector interface can generate two types of status reports, on
current crawl status, and four types of history reports, on past crawl
history.

\subsection{Status Reports}

The two types of status report are:

\begin{itemize}

\item Document Status, which lets you find information on individual
documents currently part of a job.

\item Queue Status, which lets you aggregate information about groups
of documents currently part of a job.

\end{itemize}

\subsubsection{Document Status}

This report was generated by selecting ``Sharepoint Repository
Connection,'' selecting the document state ``Documents processed at
least once,'' selecting all possible document statuses, clicking
continue, selecting ``Sharepoint Job,'' and clicking Go.

\includegraphics[width=300pt]{shp-document-report}

\begin{itemize}

\item \textbf{Connection:} The repository connection from which to 
generate a report. You must select the repository connection and click
Continue to see all repository-specific options. %Are there any?

\item \textbf{Time offset from now:} Defaults to zero. Allows you to
see estimates of future status or, with negative numbers, a record of
past status.

\item \textbf{Document state:} Allows you to select documents that
have not yet been processed or documents that have been processed
at least once.

\item \textbf{Document status:} Allows you to choose one or more 
statuses of document to report on. The statuses you can choose are:

\begin{itemize}

\item Documents that are no longer active

\item Documents currently in progress

\item Documents currently being expired

\item Documents currently being deleted

\item Documents currently available for processing

\item Documents currently available for expiration

\item Documents not yet processable

\item Documents not yet expirable

\end{itemize}

\item \textbf{Document identifier match:} A regular expression allowing
you to see only documents with matching identifiers.

\item \textbf{Jobs:} The job or jobs for which you want to generate
a report.

\end{itemize}

You can sort this report by any of the returned fields; to do so,
click the field names.

\subsubsection{Queue Status}

This report was generated by selecting ``Sharepoint Repository
Connection,'' selecting both document states, selecting all possible
document statuses, clicking continue, selecting ``Sharepoint Job,''
setting the identifier class description to \linebreak \texttt{\^{}/(.*)//}, and
clicking Go.

\includegraphics[width=300pt]{shp-queue-report}

This form offers the same fields as the Document Status report with
one addition, the \textbf{Identifier class description}, which allows
you to group results based on a regular expression. In this case,
documents are grouped together if they have the same subsite and
library path. The ungrouped documents are all analyzed together in the
first row of the table. If the identifier class description were just
\texttt{(.*)} then each document would be shown individually as a group of
one.



\subsection{History Reports}

The four types of history report are:

\begin{itemize}

\item Simple History, which lets you list an ordered set of log events
based on chosen criteria

\item Maximum Activity, which lets you see the period of time in
which a certain event happened most often

\item Maximum Bandwith, which lets you see the period of time in
which the most bandwidth was used 

\item Result Histogram, which provides log information that would be
appropriate for constructing a histogram or other diagram

\end{itemize}

Each of these reports allows you to specify a connection, one or more
activities, a start time, an end time, an entity match, and a result code
match.  Some also allow you to specify an identifier class description
and a sliding window size. This section will show sample results for
each type of report and an explanation of the fields selected.

\subsubsection{Simple History}

This report was generated by selecting ``Sharepoint Repository Connection,'' 
clicking Continue, selecting ``document ingest,'' and clicking Go.

\includegraphics[width=300pt]{shp-simple-history-report}

\begin{itemize}

\item \textbf{Connection:} The repository connection from which to generate
a report.

\item \textbf{Activities}: What crawler activities you would like to
see.  Your options are document ingest, document remove, fetch, job
abort, job continue, job end, job start, and job wait.

\item \textbf{Start time}: The earliest time in the crawler logs to be
considered for this query.  Choose ``Not specified'' for any field to
start at the beginning of the crawler's logs.

\item \textbf{End time:} The latest time in the crawler logs to be
considered for this query. Choose ``Not specified'' for any field 
to end at the current time.

\item \textbf{Entity match:} A regular expression to limit the
Identifier field. If the entity match field in the example above had
been \texttt{\^{}17}, only document fetches with Identifiers starting
with 17 would be shown.

\item \textbf{Result code match:} A regular expression to limit the
ResultCode field.

\end{itemize}

You can sort the history report by any of the returned fields; to do so,
click the field names.

\note{If you are not familiar with regular expressions,
there are a variety of tutorials available on the web,
including \url{http://gnosis.cx/publish/}\linebreak
\url{programming/regular_expressions.html} and
\url{http://perldoc.}\linebreak\url{perl.org/perlrequick.html}. If you
still have difficulty with these settings, please contact Customer Support
(see page \pageref{SupportContact}).}

\subsubsection{Maximum Activity}

This report was generated by selecting ``Sharepoint Repository
Connection,'' clicking Continue, selecting ``document ingest,''
changing the Entity match and Identifier class description, and
clicking Go.

\includegraphics[width=300pt]{shp-maximum-activity-report}

This form offers two more fields than the previous form:

\begin{itemize}

\item \textbf{Identifier class description:} A regular expression
that determines how to group identifiers together. If this were set to
\texttt{(.*)}, there would be no grouping, and so there would be only one
ingestion event per document. If this were set to \texttt{\^{}17},
then all documents with identifiers beginning with 17 would be grouped
together. The setting in the example, \texttt{()}, groups all
documents together. Some other possibilities:

\begin{itemize}

\item \texttt{(http://(.*)/)}: Groups of documents based on the site,
library, and folder path.

\item \texttt{example}: One group of documents whose identifier contains
the string ``example''.

\item \texttt{(.txt|.html)\$}: Two groups of documents sorted by file
extension.

\end{itemize}

For a document ingested with the Sharepoint Connector, the document identifier
is its Sharepoint URL.

\item \textbf{Sliding window size}: The search interval in minutes.

\end{itemize}

The report returned will have only one result per group with one or more
documents in it, if there is a clear highest activity rate, or a list of
all the results tied for highest activity rate if there are more than one.

\subsubsection{Maximum Bandwidth}

This report was generated by selecting ``Sharepoint Repository
Connection,'' clicking Continue, selecting ``document ingest,''
changing the Entity match and Identifier class description, and
clicking Go.

\includegraphics[width=300pt]{shp-maximum-bandwidth-report}

This form offers the same fields as the maximum activity form, and
returns similar results; instead of tracking events per time window,
it tracks the window with the highest average bandwith usage, measured
in bytes per second. Again, the identifier class description has been
changed to a regular expression that will match all identifiers (and
thus in this case all documents).

\subsubsection{Result Histogram}

This report was generated by selecting ``Sharepoint Repository
Connection,'' clicking Continue, selecting ``document ingest,''
altering the identifier class description to group documents based on
their file path, and clicking Go.

\includegraphics[width=300pt]{shp-activity-result-report}

This form adds one new field:

\begin{itemize}

\item \textbf{Result code class description:} A regular expression that
determines how to group result classes together; like Identifier class
descriptions but for result classes.

\end{itemize}

This report does not produce an actual histogram, but provides data that
could be used to generate histograms.  

\end{changemargin}
