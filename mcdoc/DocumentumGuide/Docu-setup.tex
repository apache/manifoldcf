% Licensed to the Apache Software Foundation (ASF) under one or more
% contributor license agreements. See the NOTICE file distributed with
% this work for additional information regarding copyright ownership.
% The ASF licenses this file to You under the Apache License, Version 2.0
% (the ``License''); you may not use this file except in compliance with
% the License. You may obtain a copy of the License at
%
% http://www.apache.org/licenses/LICENSE-2.0
%
% Unless required by applicable law or agreed to in writing, software
% distributed under the License is distributed on an ``AS IS'' BASIS,
% WITHOUT WARRANTIES OR CONDITIONS OF ANY KIND, either express or implied.
% See the License for the specific language governing permissions and
% limitations under the License.

The Documentum connector uses a special configuration file located at
\dirpath{/etc/documentum.dmcl.ini} on the GTS appliance.  In many cases,
it will be possible to simply copy the \dirpath{dmcl.ini} file from an
appropriate Webtop server onto your appliance. The appliance uses DFC
version 5.3.5; if the Webtop server uses a different version, this may not
work. In any case, the configuration file may need additonal editing. For
example, you may need to alter your configuration file if your server uses
special routing or security features. Changing your configuration may also
help if the appliance is having trouble connecting or performance is poor.

If you need to make changes, ask your Documentum administrator for
assistance with editing the configuration file.  The configuration of
the GTS appliance should be similar to that of other devices, such as
Webtop servers, that connect directly to the Documentum server.

If you do not have access to the \dirpath{dmcl.ini} file from a
Webtop server, you can configure the Documentum connector from
the appliance command line with sudo access. (If you do not have
sufficient access privileges to run this command, contact your
appliance administrator.)  First, you need to find out the hostname or
IP address of the Documentum server, or ``docbroker'', to which you will
connect. In this document, your Documentum server will be assumed to be
\url{dctmsrvr.example.com}. You will also need to know the connection
port for the Documentum server. From the appliance command line, you
should run:

\begin{consolewide}
metacarta:\~{}\$ sudo metacarta-setupdocumentum dctmsrvr.example.com port
\end{consolewide}

If your Documentum docbroker uses the default connection port 1489,
you do not need to include the optional \field{port} argument.
